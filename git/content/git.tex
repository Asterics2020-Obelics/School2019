
\section{Git}
\headlineframe{
    \includegraphics[width=0.7\textwidth]{logos/git.pdf}
}
\begin{frame}
    \centering
    \includegraphics[width=0.7\textwidth]{logos/git.pdf}

    \vspace{1em}

    \begin{itemize}
      \item Created by Linus Torvalds in 2005 for the \href{https://github.com/torvalds/linux}{Linux Kernel}
      \item Most widely used VCS in FOSS
      \item Distributed, allows offline usage
      \item Much better branching model than precursors like SVN, more \hyperlink{sec:comparison}{later}
    \end{itemize}
\end{frame}

\headlineframe{The Git Repository}
\begin{frame}{Central Concept: Repository}
  \begin{itemize}
    \item \texttt{git init} creates a git repository in the current working directory
    \item All git data is stored in the \texttt{.git} directory.
    \item Git has three different areas, changes can reside in:
  \end{itemize}

  \centering
  \begin{tikzpicture}[
      line width=1.5,
      gitstep/.style={
        draw,
        rounded corners,
        thick,
        minimum width=4cm,
        minimum height=1.2cm,
      },
    ]
    \node (wd) at (0, 0) [gitstep, fill=red!20, visible on=<1->] {Working directory};
    \node (idx) [gitstep, fill=yellow!20, below=0.4cm of wd] {Staging};
    \node (hist) [gitstep, fill=green!20, below=0.4cm of idx] {History};
    \node [right=1cm of wd, text width=0.5\textwidth, align=flush left] {
      What actually is on disk in the current working directory. 
    };
    \node [right=1cm of idx, text width=0.5\textwidth, align=flush left] {
      Changes that are saved to go into the next commit.
    };
    \node [right=1cm of hist, text width=0.5\textwidth, align=flush left] {
      The history of the project. All changes ever made. A Directed Acyclic Graph of commits.
    };
  \end{tikzpicture}
\end{frame}

\begin{frame}{Central Concept: Repository}

  \vspace{3em}
  \centering
  \begin{tikzpicture}[
      line width=1.5,
      gitstep/.style={
        draw,
        rounded corners,
        thick,
        minimum width=4cm,
        minimum height=1.2cm,
      },
    ]
    \node (wd) at (0, 0) [gitstep, fill=red!20, visible on=<1->] {Working directory};
    \node (idx) [gitstep, fill=yellow!20, visible on=<1->, below=0.4cm of wd] {Staging};
    \node (hist) [gitstep, fill=green!20, visible on=<1->, below=0.4cm of idx] {History};
    \draw[thick,->, visible on=<2->] (wd.east) to[out=0,in=0] node[right] {\texttt{git add}} (idx.east);
    \draw[thick,->, visible on=<3->] (idx.west) to[out=180,in=180] node[left] {\texttt{git commit}} (hist.west);
  \end{tikzpicture}
\end{frame}

\begin{frame}{Remotes}
  Remotes are central places, e.g. servers, where repositories can be saved and which can be used to synchronize different clients.
  \begin{center}
  \begin{tikzpicture}[line width=1.5]
    \node (hist) [draw,rounded corners,thick,minimum width=4cm,minimum height=1.2cm,fill=green!20] {History};
    \node (remote) [below=0.4cm of hist,draw,rounded corners,thick,minimum width=4cm,minimum height=1.2cm,fill=blue!20] {Remote};
    \draw[thick,->] (hist.east) to[out=0,in=0] node[right] {\texttt{git push}} (remote.east);
    \draw[thick,->] (remote.west) to[out=180,in=180] node[left] {\texttt{git pull}} (hist.west);
  \end{tikzpicture}
  \end{center}

  The main remote is canonically named \texttt{origin}.

  \texttt{git remote add <name> <url>}, e.g. \\
  \texttt{git remote add origin https://github.com/maxnoe/myrepo}
\end{frame}



\begin{frame}{Full Picture}

  \vspace{3em}
  \centering
  \begin{tikzpicture}[
      line width=1.5,
      gitstep/.style={
        draw,
        rounded corners,
        thick,
        minimum width=4cm,
        minimum height=1.2cm,
      },
    ]
    \node (wd) at (0, 0) [gitstep, fill=red!20, visible on=<1->] {Working directory};
    \node (idx) [gitstep, fill=yellow!20, visible on=<1->, below=0.4cm of wd] {Staging};
    \node (hist) [gitstep, fill=green!20, visible on=<1->, below=0.4cm of idx] {History};
    \node (remote) [below=0.4cm of hist,draw,rounded corners,thick,minimum width=4cm,minimum height=1.2cm,fill=blue!20] {Remote};

    \draw[thick,->] ([yshift=-0.15cm] wd.east) to[out=0,in=0] node[right] {\texttt{git add}} (idx.east);
    \draw[thick,->] ([yshift=-0.15cm] idx.west) to[out=180,in=180] node[left] {\texttt{git commit}} ([yshift=0.15cm]hist.west);

    \draw[thick,->] ([yshift=0.15cm] hist.east) to[out=0, in=0, looseness=3] node[right]{\texttt{git checkout}} ([yshift=0.15cm]wd.east);
    \draw[thick,->] ([yshift=0.15cm] idx.west) to[out=180, in=180] node[left]{\texttt{git reset}} (wd.west);

    \draw[thick,->] ([yshift=-0.15cm] hist.east) to[out=0,in=0] node[right] {\texttt{git push}} (remote.east);
    \draw[thick,->] (remote.west) to[out=180,in=180] node[left] {\texttt{git pull}} ([yshift=-0.15cm]hist.west);
  \end{tikzpicture}

\end{frame}


\begin{frame}{History}
  \only<1>{
    \begin{tikzpicture}
      \graph [
        grow right=1.5cm,
        nodes={
          blue!70!black,
          node font=\ttfamily,
        },
      ]{
        "ICRC 2017" [visible on=<4->, green!60!black, x=6];
        a <- b <- c <- d <- master [vertexDarkRed];
        f[visible on=<2>] ->[visble on=<2>] -> a;
        "ICRC 2017"[visible on=<2>] ->[visible on=<3>] c;
      };
    \end{tikzpicture}
  }
  \only<2>{
    \begin{tikzpicture}
      \graph [
        grow right=1.5cm,
        nodes={
          blue!70!black,
          node font=\ttfamily,
        },
      ]{
        "ICRC 2017" [visible on=<4->, green!60!black, x=6];
        a <- b <- c <- {
          d <- master [vertexDarkRed],
          e <- foo [vertexDarkRed]
        };
        "ICRC 2017"[visible on=<3>] ->[visible on=<3>] c;
      };
    \end{tikzpicture}
  }
  \only<3->{
    \begin{tikzpicture}
      \graph [
        grow right=1.5cm,
        nodes={
          blue!70!black,
          node font=\ttfamily,
        },
      ] {
        "ICRC 2017" [visible on=<4->, green!60!black, x=6],
        a <- b <- c <- d <- {e <- f, g [x=0]} <- h <- {i <-[visible on=<2->] master [vertexDarkRed, visible on=<3->], j <-[visible on=<2->] foo [vertexDarkRed, visible on=<3->]};
        "ICRC 2017" ->[visible on=<4->] e ;
      };
    \end{tikzpicture}
  }

  \vspace{1em}
  \begin{itemize}
    \item<1-> \textcolor{blue}{Commit}: State/Content at a given time
      \begin{itemize}
        \item Contains a commit message to describe the changes
        \item Commits always point to their parent(s)
        \item Commits are identified by a hash of the content, message, author(s), parent(s)
      \end{itemize}
    \item<2-> \textcolor{vertexDarkRed}{Branch}: A named pointer to a commit
      \begin{itemize}
        \item Development branches
        \item Main branch: \texttt{master}
        \item Moves to the next child if a commit is added
      \end{itemize}
    \item<4-> \textcolor{green!60!black}{Tag}: Fixed, named pointer to a commit
      \begin{itemize}
        \item For important revisions, e.g. release versions or version used for a certain paper
      \end{itemize}
  \end{itemize}
\end{frame}

\begin{frame}{Typical single-branch workflow}
  \begin{enumerate}
    \item[If new] Create or clone repository : \texttt{git init}, \texttt{git clone <url>}
    \item[If exists] \texttt{git pull}
    \item Work
      \begin{enumerate}
        \item Edit files and build/test
        \item Add changes to the next commit: \texttt{git add}
        \item Save added changes in the history as \emph{commit}: \texttt{git commit}
      \end{enumerate}
    \item Download commits that happend in the meantime: \texttt{git pull}
    \item Upload your own: \texttt{git push}
  \end{enumerate}
\end{frame}

\begin{frame}{\texttt{git init}, \texttt{git clone}}
  \begin{tabu}{>{\ttfamily}l X[,L]}
    git init          & Creates a new git repo in the CWD \\
    git clone <url>   & Clones (downloads) the repo from url \\
    rm -rf .git       & Deletes all traces of git from the repository
  \end{tabu}
\end{frame}

\begin{frame}{\texttt{git status}, \texttt{git log}}
  \begin{tabu}{>{\ttfamily}l X[,L]}
    git status & Shows current state of the repository (New, changed, deleted, renamed, untracked files) \\
    git log    & List the commits of the current branch
  \end{tabu}
\end{frame}

\begin{frame}{\texttt{git add}, \texttt{git mv}, \texttt{git rm}, \texttt{git reset}}
  \begin{tabu}{>{\ttfamily}l X[,L]}
    git add <file> … & Add files to the staging \\
    git add -p …     & Powerfull tool to only add parts of a file \\
    git mv           & like \texttt{mv}, stages automatically \\
    git rm           & like \texttt{rm}, stages automatically \\
    git reset <file> & Removes changes/files from the staging area
  \end{tabu}
\end{frame}

\begin{frame}{\texttt{git diff}}
  \begin{tabu}{>{\ttfamily}l X[,L]}
    git diff                     & Show difference between CWD and staging\\
    git diff --staged            & Show difference between staging and last commit \\
    git diff <commit1> <commit2> & Show difference between two commits
  \end{tabu}
\end{frame}

\begin{frame}{\texttt{git commit}}
  \begin{tabu}{>{\ttfamily}l X[,L]}
    git commit                       & Create a new commit from the changes in the stagin area, opens your favourite editor to compose the commmit message \\
    git commit -m "\textit{message}" & Create a new commit giving the message on the commandline \\
    git commit --amend               & Change the last commit (Adds staging to last commit, message editable)
  \end{tabu}
  \alert{\bfseries Never change commits that are already pushed}

  \begin{itemize}
    \item Style guide for commits
      \begin{itemize}
        \item First line is title/summary for the commit and should be < 60 characters
        \item Followed by one empty line
        \item Longer description of the changes, e.g. using bullet points.
      \end{itemize}
    \item Commits should be small, logical units
      \begin{itemize}
        \item \texttt{git add -p} very handy
      \end{itemize}
    \item \enquote{Commit early, commit often}
  \end{itemize}
\end{frame}

\begin{frame}{\texttt{git pull}, \texttt{git push}}
  \begin{tabu}{>{\ttfamily}l X[,L]}
    git fetch  & Download commits from the remote \\
    git pull   & Download commits and merge current branch with the remote \\
    git push   & Upload commits
  \end{tabu}
\end{frame}

\begin{frame}{\texttt{git checkout}}
  \begin{tabu}{>{\ttfamily}l X[,L]}
    git checkout <commit> & Load a certain commit from the history into the CWD (check with \texttt{git log}) \\
    git checkout <file>   & Reset \texttt{<file>} to the last commit, throwing any changes away
  \end{tabu}
\end{frame}

\begin{frame}[t]{Working using multiple branches – GitHub Workflow}
  There are multiple models of working together with git using branches

  Simplest and most popular: \enquote{GitHub-Workflow}

  \begin{itemize}
    \item Nobody directly commits into the \texttt{master} branch
    \item For each new feature / change / bugfix a new branch is created
    \item Branches should be rather shortlived
    \item Merge into master as soon as possible, then delete branch
    \item Master should always contain a working version
  \end{itemize}
\end{frame}

\begin{frame}{Branches}
  \begin{tabu}{>{\ttfamily}l X[,L]}
    git branch <name> & Create a new branch pointing to the current commit\\
    git checkout <name>   & Switch to branch \texttt{<name>} \\
    git checkout -b <name> & Create a new branch and change to it \\
    git merge <other> & Merge the changes of branch <other> into the current branch
  \end{tabu}
\end{frame}

\begin{frame}[fragile]{Beware: Merge conflicts}
  Happens when git can't merge automatically, e.g. two people edited the same line.

  \begin{enumerate}
    \item Open the files with conflicts
    \item Find the lines with conflicts and resolve by manually editing them
      \begin{verbatim}
        <<<<<<< HEAD
        foo
        ||||||| merged common ancestors
        bar
        =======
        baz
        >>>>>>> Commit-Message
\end{verbatim}
    \item Commit merged changes:
      \begin{enumerate}
        \item \texttt{git add …}
        \item \texttt{git commit}
      \end{enumerate}
  \end{enumerate}
  useful: \texttt{git config --global merge.conflictstyle diff3}
\end{frame}

\begin{frame}{\texttt{git stash}}
  \begin{tabu}{>{\ttfamily}l X[,L]}
    git stash     & Reset CWD to last commit but save the changes in the \enquote{stash} \\
    git stash pop & Get the saved changes back
  \end{tabu}
\end{frame}

\begin{frame}[fragile]{\texttt{.gitignore}}
  Many files or filetypes should not be put under version control
  \begin{itemize}
    \item Compilation results
    \item Files reproducibly created by scripts
    \item Config files containing credentials
    \item ...
  \end{itemize}

  \begin{itemize}
  \item \texttt{.gitignore} in the base of a repository
  \item One file or glob pattern per line for files that git should ignore
  \end{itemize}
  Example:
  \begin{lstlisting}[language=, keywordstyle={}]
    build/
    *.so
    __pycache__/
  \end{lstlisting}

  Github provides a default \texttt{.gitignore} for most programming languages: \href{https://github.com/github/gitignore}{github.com/github/gitignore}
\end{frame}

\begin{frame}[fragile]{Global \texttt{.gitignore}}
  For some files it makes sense to ignore them globally, for every repository

  \lstinline+git config --global core.excludesfile $HOME/.gitignore+

  E.g. strange macOS files or editor backup files

  \begin{lstlisting}[language=, keywordstyle={}]
    __MACOSX   # weird mac directory
    .DS_STORE  # weird mac metadata file
    *.swp      # vim backup files
    *~         # nano / gedit backup files
  \end{lstlisting}

\end{frame}

